\documentclass{beamer}

%Portuguese-specific commands
%--------------------------------------
\usepackage[portuguese]{babel}
%--------------------------------------
\definecolor{links}{HTML}{ECF542}
\hypersetup{colorlinks,linkcolor=,urlcolor=links}

%Hyphenation rules
%--------------------------------------
\usepackage{hyphenat}
\hyphenation{mate-mática recu-perar}
%-------------------------------------

%-------------------------------------
\graphicspath{{images/}}

\usetheme{metropolis}
\metroset{background=dark}
\metroset{block=fill}

% set captions with numbers
\setbeamertemplate{caption}[numbered]

\title{Mercado Internacional de Engenharia de Software}
\date{\today}
\titlegraphic{\hfill\includegraphics[height=2.5cm]{logo.png}}

\author{Marcos Benevides}

\begin{document}

  \maketitle

  \begin{frame}{Sumário}
    \setbeamertemplate{section in toc}[sections numbered]
    \tableofcontents
  \end{frame}

  %-------------------------------------
  \section{Apresentação}

  \begin{frame}{Apresentação}
	\begin{block}{CV}
        \begin{itemize}[<+->]
            \item Me formei em Ciência da Computação pela Universidade Federal do Maranhão (2019).
            \item \texttt{Software Development Engineer II} na empresa americana \texttt{Divisions Maintenence Group}, onde mantenho backends em F\#.
            \item Outros interesses:
                $$
                \begin{Bmatrix}
                    \texttt{Álgebra} & \texttt{Idris} & \texttt{Haskell} \\
                    \texttt{Lógica(s)} & \texttt{Métodos Formais} & \texttt{NixOS} \\
                    \texttt{SQL} & \texttt{Teoria dos Tipos} & \cdots
                \end{Bmatrix}
                $$
        \end{itemize}
	\end{block}

    \pause[\thebeamerpauses]

    Previamente eu fui \texttt{Lead Devops Engineer} na empresa brasileira \texttt{Datarisk}
  \end{frame}

  %-------------------------------------
  \begin{frame}{Apresentação}

    \begin{alertblock}{ALERTA}
        A palestra é sobre o mercado internacional, mas eu trabalho num nicho desse mercado. Então existe, sim, um viés.

        Porém, tentei destilar alguns princípios genéricos.
    \end{alertblock}

  \end{frame}

  %-------------------------------------
  \section{Recomendações}
  \subsection{Anos 60-80}

  %-------------------------------------
  \begin{frame}{Livros}
    \begin{columns}
        % Column 1
        \begin{column}{0.5\textwidth}
            \begin{itemize}
                \item<2-> É um livro antigo, contém vários exemplos difíceis de se aplicar hoje, mas é um ótimo livro para se extrair lições sobre gerência de projetos.
                \item<3-> \texttt{Lei de Brooks}, adicionar mais mão de obra num projeto que está atrasado acaba complicando mais o prazo de entrega.
            \end{itemize}
        \end{column}
        % Column 2    
        \begin{column}{0.5\textwidth}
            \begin{figure}
            \centering
                \includegraphics<1-2>[width=0.8\textwidth]{mmm00.png}
                \includegraphics<3>[width=0.8\textwidth]{mmm01.png}
                \caption{The Mythical Man-Month (1975)}
            \end{figure}
        \end{column}
    \end{columns}
  \end{frame}

  %-------------------------------------
  \begin{frame}{Xerox Parc}
    \begin{columns}
        % Column 1
        \begin{column}{0.5\textwidth}
            \begin{itemize}[<+->]
                \item<2-> Mouse e teclado
                \item<3-> Janelas
                \item<4-> Linkagem dinâmica
                \item<5-> Editores de Texto
                \item<5-> Hypertexto
                \item<5-> Hypermedia
                \item<6-> Edição colaborativa de texto em tempo real
                \item<6-> Video conferência
                \item<7-> $\ldots$
            \end{itemize}
        \end{column}
        % Column 2
        \begin{column}{0.5\textwidth}
            \begin{figure}
            \centering
                \includegraphics<1>[width=0.8\textwidth]{tmoad00.jpg}
                \includegraphics<2>[width=0.8\textwidth]{tmoad01.jpg}
                \includegraphics<3-4>[width=0.8\textwidth]{tmoad02.jpg}
                \includegraphics<5>[width=\textwidth]{tmoad03.png}
                \includegraphics<6>[width=\textwidth]{tmoad04.jpeg}
                \includegraphics<7>[width=\textwidth]{tmoad05.png}
                \caption{\href{https://www.youtube.com/watch?v=yJDv-zdhzMY}{Douglas Engelbart - The Mother of all Demos (1968)}}
            \end{figure}
        \end{column}
    \end{columns}
  \end{frame}

  %-------------------------------------
  \begin{frame}{Xerox Parc}
    \begin{columns}
        % Column 1
        \begin{column}{0.5\textwidth}
            \begin{itemize}[<+->]
                \item<2-> Computador pessoal (Xerox Alto)
                \item<2-> Dynabook (Laptop/Tablet)
                \item<3-> GUIs
                \item<4-> Smalltalk (OOP)
                \item<5-> Ethernet
                \item<5-> Email
                \item<6-> Impressora a laser
                \item<7-> Computação distribuída
                \item<7-> $\ldots$
            \end{itemize}
        \end{column}
        % Column 2
        \begin{column}{0.5\textwidth}
            \begin{figure}
            \centering
                \includegraphics[width=0.6\textwidth]{xerox_park.png}
                \caption{Dealers of Lightning: Xerox Parc and the dawn of the computing age}
            \end{figure}
        \end{column}
    \end{columns}
  \end{frame}

  %-------------------------------------
  \begin{frame}{Xerox Parc}
    \begin{columns}
        % Column 1
        \begin{column}{0.5\textwidth}
            \begin{itemize}[<+->]
                \item<2-> Como diretor do escritório de Técnicas de Processamento de Informações da ARPA, Taylor financiou o experimento de Engelbart.
                \item<3-> Taylor então contratou o pioneiro em redes Larry Roberts para supervisionar o projeto ARPAnet, um ancestral importante da Internet.
            \end{itemize}
        \end{column}
        % Column 2
        \begin{column}{0.5\textwidth}
            \begin{figure}
            \centering
                \includegraphics[width=0.7\textwidth]{bob_taylor.jpeg}
                \caption{Robert "Bob" \, Taylor}
            \end{figure}
        \end{column}
    \end{columns}
  \end{frame}

  %-------------------------------------
  \subsection{...}
  %\begin{frame}{90's}
  %  \begin{columns}
  %      % Column 1
  %      \begin{column}{0.5\textwidth}
  %          \begin{figure}
  %          \centering
  %              \includegraphics[width=0.7\textwidth]{uhhb.png}
  %              \caption{The UNIX Haters Handbook}
  %          \end{figure}
  %      \end{column}
  %      % Column 2    
  %      \begin{column}{0.5\textwidth}
  %          \begin{figure}
  %          \centering
  %              \includegraphics[width=0.6\textwidth]{pos.png}
  %              \caption{Patterns of Software: Tales from the Software Community}
  %          \end{figure}
  %      \end{column}
  %  \end{columns}
  %\end{frame}

  %-------------------------------------
  \begin{frame}{Como sobreviver num projeto coringado}
    \begin{columns}
        % Column 1
        \begin{column}{0.5\textwidth}
            \begin{itemize}[<+->]
                \item<2-> Todos que trabalham com tecnologia já presenciaram (ou vão presenciar) um projeto desses.
                \item<3-> Yourdon foca num ponto geralmente ignorado, a política, e oferece alguns insights sobre como lidar com projetos quase impossíveis e as decisões irracionais que levam às suas criações.
            \end{itemize}
        \end{column}
        % Column 2    
        \begin{column}{0.5\textwidth}
            \begin{figure}
            \centering
                \includegraphics[width=\textwidth]{dm.png}
                \caption{Death March (2002)}
            \end{figure}
        \end{column}
    \end{columns}
  \end{frame}

  %-------------------------------------
  \begin{frame}{Bolhas}
    \begin{columns}
        % Column 1
        \begin{column}{0.5\textwidth}
            \begin{itemize}[<+->]
                \item<2-> Melhor livro para entender crypto.
                \item<3-> Razão para o successo dos primeiros e-commerces, como o e-bay.
                \item<4-> Indiretamente responsável pelo crescimento do Paypal e pagamentos online.
            \end{itemize}
        \end{column}
        % Column 2    
        \begin{column}{0.5\textwidth}
            \begin{figure}
            \centering
                \includegraphics[width=0.8\textwidth]{tgbbb.png}
                \caption{The Great Beanie Baby Bubble}
            \end{figure}
        \end{column}
    \end{columns}
  \end{frame}

  %-------------------------------------
  \section{Networking}
  \subsection{Faculdade}
  \begin{frame}{Faculdade + Comunidades}
    \begin{columns}
        % Column 1
        \begin{column}{0.5\textwidth}
            \begin{figure}
            \centering
                \includegraphics[width=0.8\textwidth]{university.png}
                \caption{\textit{Sapientia ducet ad astra}}
            \end{figure}
        \end{column}
        % Column 2    
        \begin{column}{0.5\textwidth}
            \begin{figure}
            \centering
                \includegraphics[width=\textwidth]{pugma.png}
                \caption{ Python User Group - MA (2017?)}
            \end{figure}
        \end{column}
    \end{columns}

  \end{frame}

  \subsection{Open Source}
  \begin{frame}{Open Source + Comunidades}
    \begin{columns}
        % Column 1
        \begin{column}{0.5\textwidth}
            \begin{figure}
            \centering
                \includegraphics[width=0.7\textwidth]{nixos.png}
                \caption{\href{https://nixos.org/}{Nix /\ NixOS}}
            \end{figure}
        \end{column}
        % Column 2    
        \begin{column}{0.5\textwidth}
            \begin{itemize}[<+->]
                \item<2-> Contorna a limitação física das comunidades locais.
                \item<3-> Experiência direta do trabalho remoto.
                \item<4-> Experiência "simulada" do mercado internacional.
                \item<5-> Exposição e oportunidade de sair da zona de conforto.
            \end{itemize}
        \end{column}
    \end{columns}

  \end{frame}

  \subsection{Publicações}
  \begin{frame}{Publicações (relevantes)}
    \begin{columns}
        % Column 1
        \begin{column}{0.5\textwidth}
            \begin{figure}
            \centering
                \includegraphics[width=0.8\textwidth]{conf01.png}
                \caption{Adopting F\# on a Consultancy Project: From Zero to MVP to V0 Launch}
            \end{figure}
        \end{column}
        % Column 2
        \begin{column}{0.5\textwidth}
            \begin{figure}
            \centering
                \includegraphics[width=0.8\textwidth]{conf00.png}
                \caption{\href{https://www.youtube.com/watch?v=yJDv-zdhzMY}{Lessons from a data science startup using F\# and .Net in a developing country}}
            \end{figure}
        \end{column}
    \end{columns}
  \end{frame}

  \subsection{Inglês}
  \begin{frame}{Inglês}
    \begin{figure}
    \centering
        \includegraphics[width=0.6\textwidth]{english.png}
    \end{figure}
  \end{frame}

  \begin{frame}{Giga Pros - Club}
    \begin{figure}
    \centering
        \includegraphics[width=\textwidth]{giga.png}
        \caption{\href{https://gigaprosclub.com/}{gigaprosclub.com}}
    \end{figure}
  \end{frame}

  \section{Conclusão}
  \begin{frame}{O mercado atualmente}
    \begin{figure}
    \centering
        \includegraphics[width=0.7\textwidth]{layoffs.png}
        \caption{\href{https://layoffs.fyi}{Fonte: layoffs.fyi}}
    \end{figure}
  \end{frame}

  \begin{frame}{Perguntas?}
    Contatos no cyber-espaço:

    \begin{center}\href{https://github.com/mtrsk}{github.com/mtrsk}\end{center}
    \begin{center}\href{https://www.linkedin.com/in/schonfinkel/}{www.linkedin.com/in/schonfinkel}\end{center}
    \begin{center}\href{https://mtrsk.github.io}{mtrsk.github.io}\end{center}

  \end{frame}

\end{document}
